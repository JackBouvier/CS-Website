% Packages
\documentclass[12pt]{article}
\usepackage[margin=2.5cm]{geometry}
\usepackage{lipsum}
\usepackage{titlesec, titletoc}
\usepackage[svgnames, table]{xcolor}
\usepackage{algorithm}
\usepackage{algpseudocode}
\usepackage{mdframed}
\usepackage[T1]{fontenc}
\usepackage{amsmath,amsthm,amsfonts,amssymb,mathtools}
\usepackage[osf]{mathpazo}
\usepackage{enumitem}

% Formating setup
\footskip = 1 cm
\setlength{\parindent}{0pt}
\pdfpxdimen=1in
\parindent = 0pt
\definecolor{myBlue}{RGB}{0, 81, 255}
\titleformat{\section}[block]{\sffamily\large\bfseries}{\thesection}{.5em}{\textcolor{myBlue}
{\titlerule[1.5pt]}\\\sffamily}[\vspace*{-3mm}\textcolor{myBlue}{\titlerule[1.5pt]}]
\titleformat{\subsection}{\large\sffamily\bfseries}{\thesubsection}{0.5em}{\textcolor{Black}}
\newcounter{boxedlistcounter}
\newenvironment{pseudo}{%
  \setcounter{boxedlistcounter}{0}% <-- Add this line to reset the counter
  \mdframed[
    linecolor=black, % color of the border
    linewidth=1.5pt, % thickness of the border
    roundcorner=10pt, % radius of the corners
    innertopmargin=0.6\baselineskip, % space at the top of the box
    innerbottommargin=0.6\baselineskip, % space at the bottom of the box
  ]
  \fontsize{12pt}{14pt}\selectfont % add font size command here
  \mdseries % add font series command here
}{%
  \endmdframed%
}
\newcommand{\I}{\par\stepcounter{boxedlistcounter}\arabic{boxedlistcounter}.\hspace{5pt}}
\newcounter{boxedlistcounter2}
\newenvironment{Proof}{%
  \refstepcounter{boxedlistcounter2}%
  \mdframed[
    linecolor=black, % color of the border
    linewidth=1.5pt, % thickness of the border
    roundcorner=10pt, % radius of the corners
    innertopmargin=\baselineskip, % space at the top of the box
    innerbottommargin=\baselineskip, % space at the bottom of the box
  ]
  \fontsize{12pt}{14pt}\selectfont % add font size command here
  \mdseries % add font series command here
}{%
  \endmdframed%
}
\newcommand{\PI}{\par\textbullet\hspace{5pt}}
\setlist[itemize]{itemsep=1pt}

% Custom commands
\newcommand{\for}[1]{\textbf{for} #1 \textbf{do}}
\newcommand{\IF}[1]{\textbf{if} #1 \textbf{then}}
\newcommand{\ELSE}{\textbf{else}}
\newcommand{\return}[1]{\textbf{return} #1}
\newcommand{\assign}{ $\leftarrow$ }
\newcommand{\DEF}[2]{\textbf{def} #1(#2):}
\newcommand{\1}{\space \quad}
\newcommand{\2}{\quad \quad \quad}
\newcommand{\3}{\quad \quad \quad \quad \space}
\newcommand{\4}{\quad \quad \quad \quad \quad \quad}
\newcommand{\comment}[1]{\hfill \textit{\# #1}}

% Document start ------------------------------------------------------------------------------------
\begin{document}

% Section 1  ----------------------------------------------------------------------------------------
\section{Index-Based Lists}
\subsection{Abstract Data Types and Structures}
\begin{itemize}
    \item An abstract data type (ADT) is a specification of the desired behavior from the point of view 
    of the user of the data, leaving the implementation details to the programmer.
    \item A data structure is a concrete representation of data that is from the point 
    of view of an implementer, not a user.
\end{itemize}

\subsection{Index-Based List methods}
\begin{tabular}{|l|l|}
  \hline
  \rowcolor{myBlue}
  \color{white}{\textbf{Method}} & \color{white}{\textbf{Description}} \\
  \hline
  size() & (int) number of elements in the store \hspace{190pt} \\
  \hline
  isEmpty() & (boolean) whether the list is empty \\
  \hline
  get(i) & return element at index i \\
  \hline
  set(i,e) & replace the element at i with e and return the replaced element\\
  \hline
  add(i,e) & insert element e at index i and shift up elements with indexes >= i \\
  \hline
  remove(i) & remove and return the element at index i and shift down indexes >= i\\
  \hline
\end{tabular}

\subsection{Array-based List}
Time complexity of get(i) and set(i,e) is O(1) time, as they are independent of the size of the array (N) 
or the represented list (n) as an array-based implimentation allows for random access.
\begin{pseudo}
  \I \DEF{get}{i: int}
  \I \1\IF{$i < 0$ or $i >= n.size()$}
  \I \2 \return{"Index out of bound"}
  \I \1 \ELSE
  \I \2 \return{A[i]}
\end{pseudo}
\begin{pseudo}
  \I \DEF{set}{i: int, e: element}
  \I \1 \IF{$i < 0$ or $i >= n.size()$}
  \I \2 \return{"Index out of bound"}
  \I \1 \ELSE
  \I \2 result = A[i]
  \I \2 A[i] = e
  \I \2 \return{result}
\end{pseudo}
In an operation add(i, e), we must make room for the new element by shifting forward $n - i$ 
elements A[i], $\ldots$, A[$n - 1$]. Time complexity is O(n) in the worst case that i = $0$.
The same principles apply for remove(i).
\begin{pseudo}
  \I \DEF{add}{i: int, e: element}
  \I \1 \IF{n = N}
  \I \2 \return{"Array is full"}
  \I \1 \IF{$i < n$}
  \I \2 \for{j in [n-1, n-2, $\ldots$, i]}
  \I \3 A[j+1] \assign A[j]
  \I \1 A[i] \assign e
  \I \1 n \assign n + 1
\end{pseudo}
\begin{pseudo}
  \I \DEF{remove}{i: int}
  \I \1 \IF{$i < 0$ or $i >= n$}
  \I \2 \return{"Index out of bound"}
  \I \1 e \assign A[i]
  \I \1 \IF{$i < n - 1$}
  \I \2 \for{j in [i+1, i+2, $\ldots$, n-1]}
  \I \3 A[j] \assign A[j+1]
  \I \1 n \assign n - 1
  \I \1 \return{e}
\end{pseudo}

% Section 2  ----------------------------------------------------------------------------------------
\section{Positional lists}
ADT for a list where we store elements at “positions.”
Position models the abstract notion of place where a single object
is stored within a container data structure.
Unlike index, this keeps referring to the same entry even after
insertion/deletion happens elsewhere in the collection.

\subsection{Positional List methods}
\begin{tabular}{|l|l|}
  \hline
  \rowcolor{myBlue}
  \color{white}{\textbf{Method}} & \color{white}{\textbf{Description}} \\
  \hline
  size() & (int) number of elements in the store \hspace{155pt} \\
  \hline
  isEmpty() & (boolean) whether the list is empty \\
  \hline
  first() & returns the position of first element (null if empty) \\
  \hline
  last() & returns the position of last element (null if empty)\\
  \hline
  before(p) & return the position immediately before p (null if p is first) \\
  \hline
  after(p)  & returns the position immediately after p (null if p last)\\
  \hline
  insertBefore(p, e)   & inserts e in front of the element at position p\\
  \hline
  insertAfter(p, e)  & inserts e following the element at position p\\
  \hline
  remove(p)  & remove and return the element at position p\\
  \hline
\end{tabular}

\subsection{Singly linked Lists}
A singly linked list is a concerete data structure that consistes of a sequence of nodes. The
list is captured by a reference to the first node of the list, known as the head. Each node in 
a singly linked lists stores its element and a link to the next node. The last node stores a 
null reference instead of a link.
\begin{pseudo}
  \I \DEF{insertFirst}{e: element} \comment{O(1)}
  \I \1 Instaniate a new node x
  \I \1 Set e as element of x
  \I \1 Set x.next to point to head
  \I \1 Update head to point to x
\end{pseudo}
\begin{pseudo}
  \I \DEF{removeFirst}{e: element} \comment{O(1)}
  \I \1 Update head to point to next node
  \I \1 Delete the former first node
\end{pseudo}
\begin{pseudo}
  \I \DEF{insertBefore}{p, e} \comment{O(n)}
  \I \1 To find the predecessor of p we need to follow the links from the “head”.
\end{pseudo}

\subsection{Doubly linked lists}
A very natural way to implement a positional list is with a doublylinked list, so that it is easy/quick to find the position before.
Each Node in a Doubly Linked List stores its element, and a link to the previous and next nodes.
\begin{pseudo}
  \I \DEF{insertBefore}{p, e} \comment{O(1)}
  \I \1 Instaniate a new node x
  \I \1 Set e as element of x
  \I \1 Set x.next to point to p
  \I \1 Set x.prev to point to p.prev
  \I \1 Set p.prev.next to point to x
  \I \1 Set p.prev to point to x
  \I \1 \return{x}
\end{pseudo}
\begin{pseudo}
  \I \DEF{remove}{p} \comment{O(1)}
  \I \1 Set p.prev.next to point to p.next
  \I \1 Set p.next.prev to point to p.prev
  \I \1 \return{p.element}
\end{pseudo}
\begin{tabular}{|l|l|l|}
  \hline
  \rowcolor{myBlue}
  \color{white}{\textbf{Method}} & \color{white}{\textbf{Description}} & \color{white}{\textbf{Time}} \\
  \hline
  size() \hspace{65pt} & (int) number of elements in the store \hspace{80pt} & O(1) \hspace{30pt} \\
  \hline
  isEmpty() & (boolean) whether the list is empty & O(1)\\
  \hline
  first() & returns p of first element (null if empty) & O(1)\\
  \hline
  last() & returns p of last element (null if empty) & O(1)\\
  \hline
  before(p) & return the position  before p (null if p is first) & O(1)\\
  \hline
  after(p)  & returns the position  after p (null if p last) & O(1)\\
  \hline
  insertBefore(p, e)   & inserts e in front of the element at position p & O(1)\\
  \hline
  insertAfter(p, e)  & inserts e following the element at position p & O(1)\\
  \hline
  remove(p)  & remove and return the element at position p & O(1)\\
  \hline
\end{tabular}

\subsection{Array or Linked List implementation}
\textbf{Linked Lists}
\begin{itemize}
  \item good match to positional ADT
  \item efficient insertion and deletion
  \item simpler behaviour as collection grows
  \item modifications can be made as collection iterated over
  \item space not wasted by list not having maximum capacity
\end{itemize}
\textbf{Arrays}
\begin{itemize}
  \item good match to index-based ADT
  \item caching makes traversal fast in practice
  \item no extra memory needed to store pointers
  \item allow random access (retrieve element by index)
\end{itemize}

% Section 3  ----------------------------------------------------------------------------------------
\section{Stacks and queues}
These ADTs are forms of List with restricted insertiosn and removals
\begin{itemize} 
  \item stacks follow last-in-first-out (LIFO).
  \item queues follows first-in-first-out (FIFO).
\end{itemize}
\newpage

\subsection{Stacks}
\begin{itemize}
  \item push(e): inserts an element, e
  \item pop(): removes and returns the last inserted element
  \item top(): returns the last inserted element without removing it
  \item size(): returns the number of elements in the stack
  \item isEmpty(): indicates whether no elements are stored
\end{itemize}
\textbf{Applications:}
\begin{itemize}
  \item Keep track of a history that allows undoing such as Web browser history or undo sequence in a text editor.
  \item Chain of method calls in a language supporting recursion
  \item Context-free grammars
  \item Auxiliary data structure for algorithms
  \item Component of other data structures
\end{itemize}
\textbf{Array based implementation}\\
Array has capacity N. Add elements from left to right. A variable t keeps track of the index of the top element. space
used is O(N). Each operation takes O(1) time.
\begin{pseudo}
  \I \DEF{push}{e: element} \comment{O(1)}
  \I \1 \IF{t = N - 1}
  \I \2 \return{"Stack is full"}
  \I \1 \ELSE
  \I \2 t \assign t + 1
  \I \2 S[t] \assign e
\end{pseudo}
\begin{pseudo}
  \I \DEF{pop}{} \comment{O(1)}
  \I \1 \IF{isEmpty()}
  \I \2 \return{"Stack is empty"}
  \I \1 \ELSE
  \I \2 t \assign t -1
  \I \2 \return{S[t+1]}
\end{pseudo}
\begin{pseudo}
  \I \DEF{size}{}
  \I \1 \return{t+1}
\end{pseudo}

\subsection{Queues}
\begin{itemize}
  \item enqueue(e): inserts an element, e, at the end of the queue
  \item dequeue(): removes and returns element at the front of the queue
  \item first(): returns the element at the front without removing it
  \item size(): returns the number of elements stored
  \item isEmpty(): indicates whether no elements are stored
\end{itemize}
\textbf{Applications:}
\begin{itemize}
  \item Waiting lists, bureaucracy
  \item Access to shared resources (e.g., printer)
  \item Multiprogramming
  \item Auxiliary data structure for algorithms
  \item Component of other data structures
\end{itemize}
\textbf{Array based implementation}\\
Array has capacity N. Space used is O(N). Each operation takes O(1) time. Can wrap around an array. start: index of the front element.
end: index past the last element. size: number of stored elements. end = (start + size) mod N, so we only need to keep track of start and size.
\begin{pseudo}
  \I \DEF{enqueue}{e: element} \comment{O(1)}
  \I \1 \IF{size() = N}
  \I \2 \return{"Queue is full"}
  \I \1 \ELSE
  \I \2 end \assign (start + size()) mod N
  \I \2 Q[end] \assign e
  \I \2 size \assign size + 1
\end{pseudo}
\begin{pseudo}
  \I \DEF{dequeue}{} \comment{O(1)}
  \I \1 \IF{isEmpty()}
  \I \2 \return{"Queue is empty"}
  \I \1 \ELSE
  \I \2 e \assign Q[start]
  \I \2 start \assign (start + 1) mod N
  \I \2 size \assign size - 1
  \I \2 \return{e}
\end{pseudo}


\end{document}